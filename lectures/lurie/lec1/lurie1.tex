% !TEX root = ../../../main/aws_topandarith.tex
\newpage
\section{Jacob Lurie: Tamagawa numbers in the function field case}
\subsection{Lecture 1}


%$x^2+y^2$
%$x^2-y^2$
%$x^2-y^2$

\begin{dfn}
$q$ and $q'$ are in the same genus if they are $\simeq$ mod $N$ for all $N>0$.
\end{dfn}

If $q$ is a form over $\Z$ and $R$ a commutative ring.
	\[
	\{ A \in \gl_n(R) \colon q \circ A=q \}= O_q(R) \supseteq O_q(\R) \supseteq O_q(\Z)
	\]
a compact Lie group of dimension $n(n-1)/2$. 
	\[
	\mass(q)= \sum_{q' \text{of genus }q} \dfrac{1}{|O_{q'}(\Z)|},
	\]
where the sum is taken over equivalence classes of quadratic forms.


\begin{dfn}[Unimodular]
$q$ is unimodular if nondegenerate mod $p$ for all $p$
\end{dfn}

$x^2+y^2 \equiv (x+y)^2 \mod 2$.

Mass Formula (Unimodular Case):

$8 \mid n$
$\mass(q)=$ something else but
	\[
	\mass(q)= \sum_{q' \text{unimodular}} \dfrac{1}{|O_q(\Z)|}= \dfrac{\zeta(n/2)\zeta(2)\zeta(4)\cdots\zeta(n-2)}{\vol(S^1)\vol(S^2)\cdots\vol(S^{n-1})}
	\]


\begin{ex}
$n=8$ 
	\[
	RHS= \dfrac{1}{2^{14}\cdot3^5\cdot5^2\cdot7}
	\]
Then Mass-formula tells you there is a unique unimodular form in 3 variables.
\end{ex}


\begin{ex}
$n=32$
RHS is approximately 40,000,000. Looking at left side, this implies there exists \emph{a lot} of inequivalent unimodular forms in 32 variables. 
\end{ex}


Let $q,q'$ are in the same genus. 
$q= q' \circ A_N$ for some $A_n \in \gl_n(\Z/N\Z)$.
WLOG $\{A_N\}= A \in \gl_n(\hat{\Z})$
$\hat{\Z}= \projlim \Z/N\Z= \prod_p \Z_p$
$q= q' \circ A \Rightarrow q,q'$ are equivalent over $\Z_p$ for all $p$
Then $q,q'$ are equivalent over $\Q_p= \Z[1/p]$.


Hasse-Minkowski:
Then $q= q' \circ B$, where $B \in \gl_n(\Q)$
$q= q' \circ A= q \circ B^{-1} \circ A$
$B^{-1} \circ A \in O_q(\Q)/O_q(A^\text{fin})/O_q(\hat{\Z}))$
Want to count size of this. 

$B^{-1} \circ A \in O_q(\Q)/O_q(A)/O_q(\hat{\Z} \times \R))$


$A$ has a natural topology that makes it into a locally compact ring containing $\Q$ as a discrete subring. This induces $O_q(A)$, which has the structure of a locally compact group with discrete subgroup $O_q(\Q)$ and $O_q(\hat{Z}\times\R)$, a compact open subgroup. 

$O_q(\Q) / O_q(\A)$ acted on by $O_q(\hat{\Z}\times\R)$
	\[
	\text{\# of orbits}= \dfrac{\mu(O_p(\Q)\ O_q(\A))}{\mu(O_q(\hat{\Z}\times\R))}
	\]
Not quite correct. 

$\so_Q(A)$ has a canonical Haar measure called Tamagawa measure
	\[
	2^k \mass(q)= \dfrac{\mu(\so_q(\Q) / \so_q(A))}{\mu(\so_q(\hat{\Z} \times \R))}
	\]
$\so_q(A)= \so_q(\R) \times \prod_p^\text{res} \so_q(\Q_p)$
$V_\R$ is the space of translation invariant topological forms on $\so_q(\R)$.
$V_\R \supseteq V_\Q$ the space of translation invariant topological forms on $\so_q(\Q)$


$V_{\Q_p}$ the space of translation invariant topological forms on $\so(\Q_p)$

$\so_q(\Q_p)$ is a $p$-adic analytic Lie group. 

$0 \neq \omega \in V_\Q \mapsto \mu_{\omega,\R}$

$0 \neq \omega \in V_\Q \mapsto \mu_{\omega,\Q_p}$

Tamagawa Measure
	\[
	\mu_\text{Tam}= \prod_p \mu_{\omega,\Q_p} \times \mu_{\omega,\R}
	\]
independent of $\omega$
	\[
	\mass(q)= 2^{-k} \dfrac{\mu_\text{Tam}(\so_q(\Q)/\so_q(\R))}{\mu_\text{Tam}(\so_q(\hat{\Z}\times\R))}
	\]

$\so_q(\hat{\Z}\times\R)= \so_q(\R) \times \prod_p \so_q(\Z_p)$
$\mu_\text{Tam}(\so_q(\hat{\Z}\times\R)) \defeq \mu_{\omega,\R}(\so_q(\R)) \times \prod_p \mu_{\omega,\Q_p}(\so_q(\Z_p))$


Mass Formula (Tamagawa-Weil Version)
$\mu_\text{Tam}(\so_q(\Q) / \so_q(\A))= \Z$
$\so_q$ has a two-sheeted double cover $\spin_q$

Equivalent:
$\mu_\text{Tam}(\spin_q(\Q) / \spin_q(A))=1$

Conjecture (Weil)

Let $G$ be a simply connected semisimple algebraic group over $\Q$
$\mu_\text{Tam}(G(\Q)/G(\A))=1$, where $G(\Q)$ is $\tau_G$, the Tamagawa number of $G$.

Now a theorem, proved by Weil in many cases, Langlands when split group, \dots, 






















